\documentclass[serif,mathserif]{beamer}
\usepackage{amsmath, amsfonts, epsfig, xspace}
\usepackage{algorithm,algorithmic}
\usepackage{pstricks,pst-node}
\usepackage{multimedia}
\usepackage[normal,tight,center]{subfigure}
\setlength{\subfigcapskip}{-.5em}
\usepackage{beamerthemesplit}
\usetheme{lankton-keynote}
\usepackage{CJKutf8}

\author[Group 10]{李声涛 \quad 17214643 \quad 宋日辉 \quad 17214675 \\ 黄莉 \quad 17210000 \quad 管卓群 \quad 17210000 \\ 周晓梅 \quad 17214729}

\title[PointNet\hspace{2em}\insertframenumber/\inserttotalframenumber]{PointNet: Deep Learning on Point Sets for 3D Classification and Segmentation}

\date{December 19, 2017} %leave out for today's date to be insterted

\institute{School of Data and Computer Science, SYSU}

\begin{document}
\begin{CJK*}{UTF8}{gbsn}
  \maketitle
\end{CJK*}

% \section{Introduction}  % add these to see outline in slides

\begin{frame}
  \frametitle{Content}
  \begin{itemize}
  	\item Introduction
  	\item Problem Statement
  	\item Deep Learning on Point Sets
  	\item Experiment
  	\item Conclusion
  \end{itemize}
\end{frame}

\begin{frame}{Introduction}
	\begin{itemize}
		\item Point Set
	\end{itemize}
	\begin{figure}[t]
		\centering
		\subfigure[2D Point Set]{
			\includegraphics[width=5cm]{image/timg.jpeg}}
		\subfigure[3D Point Set (Point Cloud)]{
			\includegraphics[width=5cm]{image/point_cloud.png}}
	\end{figure}
\end{frame}

\begin{frame}{Traditional Point Cloud Processing}
	%\begin{columns}
	%	\column{0.5\textwidth}
		\begin{itemize}
			\item Edge-based methods
			\item Model-based methods
			\item Region-based methods
			\item Attributes-based methods
			\item Graph-based methods
		\end{itemize}
	%	\column[]{0.7\textwidth}
		\begin{figure}
			\includegraphics[width=6cm]{image/desk.png} 
			\includegraphics[width=2cm]{image/leg.png} 
		\end{figure}
%	\end{columns}
	
\end{frame}

\begin{frame}{Neural Network Based Methods}
	\begin{itemize}
		\item Volumetric CNNs: 3D voxel grids
		\begin{itemize}
			\item Constrained by resolution
		\end{itemize}
		\item Multi-view CNNs: collections of images
		\begin{itemize}
			\item Nontrivial to extend them to scene understanding or other 3D tasks.
		\end{itemize}
	\end{itemize}
\end{frame}

\begin{frame}{PointNet}
	\begin{itemize}
		\item A novel deep net architecture
		\item Input: point set
		\item Tasks: 3D shape classification, shape part segmentation, and scene semantic parsing 
		\item Simple, effective and robust
	\end{itemize}
	\begin{figure}
		\includegraphics[width=9cm]{image/teaser.png}
	\end{figure}
\end{frame}

\begin{frame}{Problem Statement}
	\begin{itemize}
		\item A point cloud is represented as a set of 3D points:  
		\begin{itemize}
			\item $ \{P_i | i = 1,...,n\} $
			\item $ P_i = (x, y, z) $
			\item Extra feature channels: color, normal, etc.
		\end{itemize}
	\end{itemize}
\end{frame}

\begin{frame}{object classification}
	\begin{itemize}
		\item The input: \\
		Directly sampled from a shape
		Pre-segmented from a scene point cloud. 
		\item The output: \\
		This deep network outputs k scores for all the k candidate classes.
	\end{itemize}
\end{frame}

\begin{frame}{semantic segmentation}
	\begin{itemize}
		\item The input: \\
		A single object for part region segmentation
		A sub-volume from a 3D scene for object region segmentation.
		\item The output: \\
		This model will output $ n \times m $ scores for each of the n points and each of the m semantic    
		subcategories.
	\end{itemize}
\end{frame}

\begin{frame}{Deep Learning on Point Sets}
	\begin{itemize}
		\item Properties of Point Sets
		\item PointNet Architecture
	\end{itemize}
\end{frame}

\begin{frame}{Properties of Point Sets}
	\begin{itemize}
		\item Unordered
		\item Interaction among points
		\item Invariance under transformations
	\end{itemize}
\end{frame}

\begin{frame}{PointNet Architecture}
	\begin{figure}
		\includegraphics[width=9cm]{image/pointnet_arch.jpg}
	\end{figure}
	\begin{itemize}
		\item Symmetry Function for Unordered Input
		\item Local and Global Information Aggregation
		\item Joint Alignment Network
	\end{itemize}
\end{frame}

\begin{frame}{Experiments}
	\begin{itemize}
		\item Applications
		\item Architecture Design Analysis
		\item Visualizing PointNet
		\item Time and Space Complexity Analysis
	\end{itemize}
\end{frame}

\begin{frame}{Applications-3D Object Classification}
	\begin{itemize}
		\item 12,311 CAD models
		\item from 40 man-made object categories, 
		\item split into 9,843 for training and 2,468 for testing
	\end{itemize}
	\begin{figure}
		\includegraphics[width=8cm]{image/application.png}
	\end{figure}
\end{frame}

\begin{frame}{Applications-3D Object Part Segmentation }
	\begin{itemize}
		\item ShapeNet part data set
		\item 16,881 shapes from 16 categories, annotated with 50 parts in total
		\item mIoU?
	\end{itemize}
	\begin{figure}
		\includegraphics[width=9cm]{image/seg.png}
	\end{figure}
\end{frame}

\begin{frame}{Applications-Semantic Segmentation in Scenes }
	\begin{itemize}
		\item Stanford 3D semantic parsing data set
		\item The dataset contains 3D scans from Matterport scanners in 6 areas including 271 rooms. Each point in the scan is annotated with one of the semantic labels from 13 categories (chair, table, floor, wall etc)
	\end{itemize}
	\begin{figure}
		\includegraphics[width=7cm]{image/sematic.png}
	\end{figure}
\end{frame}

\begin{frame}{Applications-Semantic Segmentation in Scenes}
	\begin{figure}
		\includegraphics[width=8cm]{image/scencs1.png}
	\end{figure}
	\begin{figure}
		\includegraphics[width=8cm]{image/scencs2.png}
	\end{figure}
\end{frame}

\begin{frame}{Architecture Design Analysis}
	\begin{itemize}
		\item Three approaches to achieve order invariance.
		\item ModelNet40 shape classification problem
	\end{itemize}
	\begin{figure}
		\includegraphics[width=9cm]{image/arch.png}
	\end{figure}
\end{frame}

\begin{frame}{Architecture Design Analysis}
	\begin{itemize}
		\item Effectiveness of Input and Feature Transformations
		\item ModelNet40 shape classification problem
	\end{itemize}
	\begin{figure}
		\includegraphics[width=7cm]{image/arch2.png}
	\end{figure}
\end{frame}

\begin{frame}{Architecture Design Analysis}
	\begin{itemize}
		\item Robustness Test
		\item ModelNet40 shape classification problem
	\end{itemize}
	\begin{figure}
		\includegraphics[width=9cm]{image/arch3.png}
	\end{figure}
\end{frame}

\begin{frame}{Visualizing PointNet}
	\begin{itemize}
		\item critical point sets $ C_S $ \& the upper-bound shapes $ N_S $
	\end{itemize}
	
	\begin{figure}
		\includegraphics[width=7cm]{image/visual.png}
	\end{figure}
\end{frame}

\begin{frame}{Time and Space Complexity Analysis}
	\begin{itemize}
		\item PointNet’s space and time, complexity is O(N)
		\item point cloud classification: 1K objects/second
		\item semantic segmentation: 2 rooms/second
		\item 1080X GPU on TensorFlow
	\end{itemize}
	\begin{figure}
		\includegraphics[width=9cm]{image/time.png}
	\end{figure}
\end{frame}

\begin{frame}{Conclusion}
	\begin{itemize}
		\item A brief introduction
		\item PointNet architecture
		\item Experiment result
	\end{itemize}
\end{frame}

\begin{frame}{Repeat the experiment}
	\begin{itemize}
		\item Tensorflow
		\item CPU: i7 - 5700
		\item GPU: Geforce 1070
		\item Training time: 2h31min(classification) and 16h28min(part segmentation)
	\end{itemize}
	\begin{figure}
		\includegraphics[width=8cm]{image/classify_log.png}
	\end{figure}
\end{frame}
\end{document}

%These are examples.
% \section{Main Body} % add these to see outline in slides
%
%\begin{frame}
%	\frametitle{Equations}
%	Equations are easy
%	\begin{itemize}
%		\item Just and paste equations\pause
%		\item From the paper!
%		\begin{equation*}
%		\textbf{p}^* = \underset{\textbf{p}}{\arg\!\min}~\sum_{\textbf{x}}\left[ I(\textbf{W}(\textbf{x};\textbf{p})) - T(\textbf{x}) \right]^2
%		\end{equation*}
%	\end{itemize}
%\end{frame}


%\begin{frame}
%	\frametitle{Pictures}
%	\begin{figure}[t]
%		\centering
%		\subfigure[First Frame]{
%			%    \includegraphics[width=3cm]{figures/naked_leaves/00000001}}
%			\includegraphics[width=3cm]{image/lion.png}}
%		\subfigure[Middle Frame]{
%			%    \includegraphics[width=3cm]{figures/naked_leaves/00000120}}
%			\includegraphics[width=3cm]{image/lion.png}}
%		\subfigure[Last Frame]{
%			%    \includegraphics[width=3cm]{figures/naked_leaves/00000240}}
%			\includegraphics[width=3cm]{image/lion.png}}
%	\end{figure}
%\end{frame}
%
%\begin{frame}
%	\frametitle{A Movie}
%	\begin{center}
%		\movie[height=5cm,width=6.5cm,poster,autostart,loop]{}{leaves.avi}
%	\end{center}
%	\begin{itemize}
%		\item Movies only seem to work in Adobe Reader
%		\item Movie file is not embedded, it must be on the computer
%	\end{itemize}
%\end{frame}
%
%% \section{Conclusion} % add these to see outline in slides
%
%\begin{frame}
%	\frametitle{Credits}
%	\begin{itemize}
%		\item Brought to you by www.shawnlankton.com
%		\item Please let me know about improvements!
%		\item This was supposed to look like a KeyNote Show
%		\item inspiration: http://www.ucl.ac.uk/~ucbpeal/latexposter.html
%		\item inspiration: http://newsgroups.derkeiler.com/... (in code)
%		%http://newsgroups.derkeiler.com/Archive/Comp/comp.text.tex/2007-11/msg00299.html
%	\end{itemize}
%\end{frame}
%
%\begin{frame}
%	\frametitle{Questions}
%\end{frame}